\documentclass[12pt,letterpaper, onecolumn]{exam}
\usepackage{amsmath}
\usepackage{amssymb}
\usepackage[lmargin=71pt, tmargin=1.2in]{geometry}  %For centering solution box
\usepackage{graphicx} % Required for inserting images
\usepackage{lastpage}

% Document's meta-data
\newcommand{\subject}{Lógica e Matemática Computacional}
\newcommand{\assignment}{Atividade 02}
\newcommand{\authorfullname}{José Augusto Queiroz Comparotto Gomes}
\newcommand{\authorname}{José A. Q. C. Gomes}
\newcommand{\authorno}{398439413098}
\newcommand{\professor}{Prof.ª Noiza Waltrick Trindade}
\newcommand{\course}{Ciência da Computação — Noturno}
\newcommand{\classno}{1309820242A}
\newcommand{\location}{Campo Grande-MS}
\newcommand{\documentdate}{22 de setembro de 2024}
\newcommand{\university}{Universidade Anhaguera — Uniderp}

\author{\authorfullname}
\title{\subject: \assignment}

% General page styling
\lhead{\subject\\}
\rhead{\assignment\\}
\lfoot{\authorname}
\cfoot{Página \thepage \ de \pageref{LastPage}}
\rfoot{RA: \authorno}

\thispagestyle{empty}   %For removing header/footer from page 1

\begin{document}

% Presenting page
\begingroup  

    \centering
    \begin{figure}
        \centering
        \includegraphics[width=0.15\linewidth]{assets/uniderp.jpg}
        \label{fig:university-logo}
    \end{figure}
    
    \rule{\textwidth}{2pt}  \\[1em]
    
    \LARGE \authorfullname

    \vfill
    
    \LARGE \subject     \\
    \LARGE \assignment
    
    \vfill
    
    \large À \professor \\

    \vfill
    
    \large \course          \\
    \large Turma \classno   \\[1em]
    
    \rule{\textwidth}{2pt}  \\[1em]

    \large \location        \\
    \large \documentdate    \\
    
    \pagebreak
\endgroup

\pointsdroppedatright   %Self-explanatory
\printanswers
\renewcommand{\solutiontitle}{\noindent\textbf{Resposta:}\enspace}  

\begin{questions}

    \question[q1] Considerando que a proposição “Nenhum homem bom pratica o mal” é falsa, qual das seguintes alternativas apresenta uma proposição verdadeira?
    
    \begin{parts}
        \part[a] Todo homem bom pratica o mal.
        \part[b] Todo homem bom não pratica o mal.
        \part[c] Alguns homens bons não praticam o mal. 
        \part[d] Pelo menos um homem bom pratica o mal.
        \part[e] Não há homem bom que pratique o mal.
    \end{parts}
    
    \begin{solution}
        (d) Pelo menos um homem bom pratica o mal.

        \textbf{Justificativa:}
        
        A negação da inexistência de \(p\) (\( \neg \ \nexists \ p \)) é equivalente à afirmação da existência de \(p\) (\( \exists \ p \)).

        Dado que \(p = \text{“homem bom pratica o mal”} \) e que \( \ \nexists \ p = F\), tem-se que: \( \exists \ p  = V\).

        A alternativa (d) “Pelo menos um homem bom pratica o mal.” é compatível com: \(\exists \ p\), portanto é uma proposição Verdadeira. 
        
    \end{solution}
    
    \question[q2] Considere a seguinte sentença: “Todo professor é bem-humorado”. A negação dessa sentença é:
    
    \begin{parts}
        \part[a] Não existe professor mal-humorado.
        \part[b] Existe professor mal-humorado.
        \part[c] Alguns professores são bem-humorados.
        \part[d] Existe professor bem-humorado.
        \part[e] Nenhum professor é mal-humorado.
    \end{parts}
    
    \begin{solution}
        (b) Existe professor mal-humorado.

        \textbf{Justificativa:}
        
        A negação da afirmação universal de \(p\) (\( \neg \forall p \)) é equivalente à afirmação da existência de não-\(p\) (\( \exists \neg p \)).

       Considerando que “bem-humorado” e “mal-humorado” têm sentidos explicitamente opostos, é válido assumir que se \(p = \text{“professor é bem-humorado”} \), então \( \neg p = \text{“professor é mal-humorado”} \).

        A alternativa (b) “Existe professor mal-humorado.” é compatível com: \( \exists \neg p \), portanto é a negação de \(\forall p\) (“Todo professor é bem-humorado”).
        
    \end{solution}

    \question[q3] Sejam as proposições, P: Marcos é alto; Q: Marcos é elegante. Traduzir para a linguagem simbólica as seguintes proposições. 
    
    \begin{parts}
        \part[a] Marcos é alto e elegante.
        \part[b] Marcos é alto, mas não é elegante. 
        \part[c] Não é verdade que marcos é baixo ou elegante.
        \part[d] Marcos não é nem alto e nem elegante.
        \part[e] Marcos é alto ou é baixo e elegante. 
        \part[f] É falso que Marcos é baixo ou que não é elegante. 
    \end{parts}
    
    \begin{solution}                            \\[0.5em]
        (a) \( P \wedge Q \)                    \hfill
        (b) \( P \wedge \neg Q \)               \hfill
        (c) \( \neg ( \neg P \vee Q ) \)        \\[0.5em]
        (d) \( \neg P \wedge \neg Q \)          \hfill
        (e) \( P \vee ( \neg P \wedge Q ) \)    \hfill
        (f) \( \neg ( \neg P \vee \neg Q ) \)   \\[1em]
        \textbf{Justificativa:} Considerando que,
        \begin{itemize}
            \item[I.] “alto” e “baixo” têm sentidos exclusivamente opostos, é válido que se \( P = \text{“Marcos é alto”} \), então \( \neg P = \text{“Marcos é baixo”} \).
            \item[II.] A fórmula “\(p\) e \(q\)” equilave a: \( p \wedge q\).
            \item[III.] A fórmula “\(p\), mas não \(q\)” equilave a: \( p \wedge \neg q\).
            \item[IV.] A fórmula “Não é verdade que \(p\)” equilave a: \( \neg p \).
            \item[V.] A fórmula “\(p\) ou \(q\)” equilave a: \( p \vee q\).
            \item[VI.] A fórmula “nem \(p\), nem \(q\)” equilave a: \( \neg p \wedge \neg q \).
        \end{itemize}
        É possível reescrever cada sentença dada em sua respectiva forma simbólica. 
    \end{solution}

    \question[q4] Construir a Tabela Verdade das seguintes proposições: \\[1em]
    (a) \( P \wedge \neg Q \Rightarrow P \) \hfill
    (b) \( \neg P \iff \neg Q \)            \hfill
    (c) \( \neg (\neg P \vee \neg Q) \)     

    \begin{solution}
        \begin{tabular}{| c | c || c | c | c |} 
         \hline
            \( P \) & \( Q \) &
            (a) \( P \wedge \neg Q \Rightarrow P \) &
            (b) \( \neg P \iff \neg Q \)  &
            (c) \( \neg (\neg P \vee \neg Q) \) \\
         \hline
         \hline
         \( F \) & \( F \) & \( V \) & \( V \) &  \( F \) \\
         \hline
         \( F \) & \( V \) & \( V \) & \( F \) &  \( F \) \\
         \hline
         \( V \) & \( F \) & \( V \) & \( F \) &  \( F \) \\
         \hline
         \( V \) & \( V \) & \( V \) & \( V \) &  \( V \) \\
         \hline
        \end{tabular}
    \end{solution}

    \pagebreak
    
    \question[q5] Dadas as proposições simples \(p\) e \(q\), tais que \(p\) é verdadeira e \(q\) é falsa, considere as seguintes proposições compostas: \\[1em] 
        (1) \( p \wedge q \) \hfill
        (2) \( \neg p \Rightarrow q \) \hfill
        (3) \( \neg ( p \vee \neg q ) \) \hfill
        (4) \( \neg ( p \iff q ) \) \\[1em] 
    Quantas dessas proposições compostas são verdadeiras? \\[1em]
        (a) nenhuma. \hfil
        (b) uma.     \hfil
        (c) duas.    \hfil
        (d) três.    \hfil
        (e) quatro.
    
    \begin{solution}
        (b) duas
        
        \textbf{Justificativa:}  Dado que \( p = V\) e que \( q = F \), tem-se que:
        
        \begin{itemize}
            \item[(1)] \( (p \wedge q) = (V \wedge F) = F\) 
            \item[(2)] \( (\neg p \Rightarrow q ) = ( \neg V \Rightarrow F ) = ( F \Rightarrow F ) = V\)
            \item[(3)] \( \neg ( p \vee \neg q ) = \neg ( V \vee \neg F ) = \neg ( V \vee V ) = \neg V = F \)
            \item[(4)] \( \neg ( p \iff q ) = \neg ( V \iff F ) = \neg F = V \)
        \end{itemize}

        Desta forma, temos que apenas as proposições (2) e (4) são verdadeiras. Portanto, a resposta é a alternativa (b) duas.

        
    \end{solution}
    
    \question[q6] Utilizando as letras proposicionais adequadas na proposição composta “Nem Antônio é  desembargador nem Jonas é juiz”, assinale a opção correspondente à simbolização correta dessa proposição:\\[1em]
        (a) \( \neg ( A \wedge B ) \)           \hfill
        (b) \( ( \neg A ) \Rightarrow B \)      \hfill
        (c) \( \neg ( A ) \vee ( \neg B ) \)    \\[1em]
        (d) \( \neg ( A \vee ( \neg B ) ) \)    \hfill
        (e) \( ( \neg A ) \wedge ( \neg B ) \)
        
    \begin{solution}
        (e) \( ( \neg A ) \wedge ( \neg B ) \)
        
        \textbf{Justificativa:}
        Em  “Nem Antônio é desembargador, nem Jonas é juiz”, podemos separar as seguintes partes: \\[1em]
            A = “Antônio é desembargador” e 
            B = “Jonas é juiz”.\\[1em]
        A fórmula “nem \(p\), nem \(q\)” corresponde à conjunção das negações: \( \neg p \wedge \neg q \).   \\[1em]
        Portanto, a proposição “Nem Antônio é desembargador, nem Jonas é juiz” corresponde a \( ( \neg A ) \wedge ( \neg B ) \).
    \end{solution}
    
    \question[q7] Sabe-se que as proposições:
        \begin{enumerate}
            \item[I] — Se Aristides faz gols, então o GFC é campeão;
            \item[II] — O Aristides faz gols ou o Leandro faz gols;
            \item[III] — Leandro faz gols;
        \end{enumerate}
    são, respectivamente, falsa, verdadeira e falsa. Daí, conclui-se que:
    
    \begin{parts}
        \part Aristides não faz gols ou o GFC não é campeão.
        \part Aristides faz gols e o GFC não é campeão. 
        \part Aristides não faz gols e o GFC é campeão.
        \part Aristides faz gols e o GFC é campeão. 
        \part Aristides não faz gols e o GFC não é campeão.
    \end{parts}

    \begin{solution}
        (b) Aristides faz gols e o GFC não é campeão.
        
        \textbf{Justificativa:}
            Dado que a proposição “I” segue a estrutura condicional “se \(p\), então \(q\)” (\( p \Rightarrow q \)), sendo \( p \) a afirmação “Aristides faz gols” e \( q \) a afirmação “GFC é campeão”, e foi dado que \( I = F \), sabemos que a condição é falsa.
            
            Pela tabela verdade de uma proposição condicional, a única situação em que essa proposição é falsa ocorre quando \( p = V \) e \( q = F \), ou seja, quando “Aristides faz gols” (\( p = V \)) e “GFC não é campeão” (\( q = F \)).
            
            Portanto, conclui-se que “Aristides faz gols” e “GFC não é campeão”, correspondendo à alternativa (b).
    \end{solution}

    \pagebreak
    
    \question[q8] Existem três suspeitos de invadir uma rede de computadores: Lucas, Mariana e José. Sabe se que a invasão foi efetivamente cometida por um ou por mais de um deles, já que podem ter agido individualmente ou não. Sabe-se ainda que:
        
        \begin{enumerate}
            \item[\(P_1\))] se Lucas é inocente, então Mariana é culpada;
            \item[\(P_2\))] ou José é culpado, ou Mariana é culpada;
            \item[\(P_3\))] José não é inocente.
        \end{enumerate}
        
    Com base nestas considerações, conclui-se que: 
    
    \begin{parts}
        \part somente Lucas é inocente.
        \part somente Mariana é culpada.
        \part somente José é culpado.
        \part são culpados Mariana e José.
        \part são culpados Lucas e José.
    \end{parts}

    \begin{solution}
        (e) são culpados Lucas e José

        \textbf{Justificativa:} Sejam, \\[0.5em]
            L = “Lucas é culpado”;   \hfill
            M = “Mariana é culpada”; \hfill
            J = “José é culpado”.    

        Considera-se que para qualquer \(x\), “\(x\) é inocente” é negação para “\(x\) é culpado(a)”. 

        É considerado verdadeiro que: \\[1em]
        \(P_0\): \( L \vee M \vee J \); \textit{(enunciado)} \hfill
        \(P_1\): \( \neg L \Rightarrow M \); \hfill
        \(P_2\): \( J \veebar M \); \hfill
        \(P_3\): \( \neg \neg J \). \\[1em]
        A dupla negação em \(P_3\) infere que \(J = V\). Portanto, “José é culpado”.

        Diante disto, infere-se a partir de \(P_2\) que \(M = F\). Ou seja, “Mariana é inocente”.

        Desta forma, a partir contra-positiva de \(P_1\): (\( \neg M \Rightarrow L \)), como \(M = F\), concluí-se que \( L = V \): “Lucas é culpado”.
    
        Portanto, a alternativa que reflete estas conclusões é a letra (e): são culpados Lucas e José.
        
    \end{solution}

    \pagebreak
    
    \question[q9] Do ponto de vista lógico, uma afirmação equivalente à afirmação: “o bolso está furado ou as moedas não caem no chão” é: 
    
    \begin{parts}
        \part o bolso não está furado e as moedas não caem no chão.
        \part se o bolso não está furado, então as moedas não caem no chão. 
        \part o bolso está furado e as moedas caem no chão. 
        \part se o bolso está furado, então as moedas caem no chão.
        \part se as moedas não caem no chão, então o bolso não está furado. 
    \end{parts}

    \begin{solution}
        (b) se o bolso não está furado, então as moedas não caem no chão.
    \end{solution}
    
    \question[q10] A negação de “Se a canoa não virar, eu chego lá” é: 
    
    \begin{parts}
        \part A canoa não vira e eu não chego lá.
        \part Se a canoa virar, eu não chego lá.
        \part Se a canoa não virar, eu não chego lá.
        \part A canoa vira e eu chego lá.
        \part Se eu não chego lá, a canoa vira.
    \end{parts}

    \begin{solution}
        (a) A canoa não vira e eu não chego lá.
    \end{solution}
    
    \question[q11] Dizer que “Alexandre foi aos Lençóis Maranhenses, se e somente se, fez sol” é logicamente equivalente dizer que: 
    
    \begin{parts}
        \part Ou Alexandre foi aos Lençóis Maranhenses, ou fez sol.
        \part Não fez sol, se e somente se, Alexandre foi aos Lençóis Maranhenses. 
        \part Se Alexandre foi aos Lençóis Maranhenses, então não fez sol.
        \part Se Alexandre foi aos Lençóis Maranhenses, então fez sol.
        \part Fez sol, se e somente se, Alexandre foi aos Lençóis Maranhenses.
    \end{parts}

    \begin{solution}
        (e) Fez sol, se e somente se, Alexandre foi aos Lençóis Maranhenses.
    \end{solution}

    \pagebreak

    \question[q12] Considere a sentença: “Se gosto de capivara, então gosto de javali”. Uma sentença logicamente equivalente à sentença dada é: 
    
    \begin{parts}
        \part Se não gosto de capivara, então não gosto de javali.
        \part Gosto de capivara e gosto de javali.
        \part Não gosto de capivara ou não gosto de javali.
        \part Gosto de javali ou não gosto de capivara.
        \part Se gosto de javali, então não gosto de capivara.
    \end{parts}

    \begin{solution}
        (d) Gosto de javali ou não gosto de capivara.
    \end{solution}
    
\end{questions}

\end{document}